\appendices
%
% If you only have one appendix, you should change the above to:
%\appendix
%

\chapter{Sage implementations of Zig-Zag and Replacement Products}
\section{Tools for working with Enumerations and the Zig-Zag and Replacement Products}
\label{app:enum}
\begin{sageblock}

# Take a given digraph G and label the edges using the 
# enumeration provided
def enumerate_graph(G,enum):
     for source_vert in enum.iterkeys():
         for term_vert in enum[source_vert].iterkeys():
             G.set_edge_label(source_vert,term_vert,\
             enum[source_vert][term_vert])
# Generate a random enumeration from a given digraph G
def random_enumeration(G):
    import random

    enum = dict()

    for v in G.vertex_iterator():
        nhbd = G[v]
        deg = len(nhbd)
        random.shuffle(nhbd)
        enum[v] = dict(zip(range(0,deg), nhbd))

    return enum

# Compute the inverse of a one-to-one dictionary
def inverse_dict(d):

    return dict( [ [v,k] for k,v in d.iteritems()])

# Starting with an enumeration and an initial edge, 
# compute the unique parity circuit
def parity_circuit(enum, init_edge):
    
    inv_enum = dict([[v, inverse_dict(d)] for v,d \
    in enum.iteritems()])
    
    init_vert = init_edge[0]
    next_vert = init_edge[1]
    init_enum = inv_enum[init_vert][next_vert]

    circ = [init_vert, next_vert]
    cur_vert = init_vert


    out_enum = inv_enum[init_edge[0]][init_edge[1]]
    in_enum = inv_enum[next_vert][cur_vert]
    out_enum = (in_enum + 2)%4
    cur_vert = next_vert

    
    while ( (cur_vert, out_enum) != (init_vert, init_enum) ):
        next_vert = enum[cur_vert][out_enum]
        circ.append(next_vert)
        in_enum = inv_enum[next_vert][cur_vert]
        out_enum = (in_enum + 2)%4
        cur_vert = next_vert
        
    return circ[:-1]
#Given two digraphs G and H with enumeration of G  and return the
# zig-zag product GzzH
def zig_zag(G,H, enum):
    ZZ = DiGraph()
    
    inv_enum = dict([[v, inverse_dict(d)] for v,d \
    in enum.iteritems()])
     
    for e in G.edge_iterator():
        [ ZZ.add_edge((e[0],a) ,(e[1],b) ) \ 
        for a in H[inv_enum[e[0]][e[1]]] \
        for b in H[inv_enum[e[1]][e[0]] ] ]
    return ZZ

# Computes the replacement product of two
# digraphs G and H with enumeration

def replacement(G,H,enum):
    RR = DiGraph()
    inv_enum = dict([[v, inverse_dict(d)] for v,d\
    in enum.iteritems()])

    for v in G.vertex_iterator():
        [ RR.add_edge((v,e[0]), (v,e[1])) for e in \
        H.edge_iterator() ]
    for e in G.edge_iterator():
        RR.add_edge( (e[0], inv_enum[e[0]][e[1]] ),\
        (e[1],inv_enum[e[1]][e[0]] )) 

    return RR
\end{sageblock}

\section{Sage implementation of Generalized Graph Object with Zig-Zag Product}
\label{app:nets}
\begin{sageblock}
#***********************************************************
#  Copyright (C) 2008 David Monarres <dmmonarres@gmail.com>
#
#  Distributed under the terms of the GNU General Public \
#  License version 2 (GPLv2)
#
#  This code is distributed in the hope that it will be useful,
#  but WITHOUT ANY WARRANTY; without even the implied 
#  warranty of MERCHANTABILITY or FITNESS FOR A PARTICULAR PURPOSE.  
#  See the GNU General Public License for more details.
#
#  The full text of the GPL is available at:
#
#                  http://www.gnu.org/licenses/
#************************************************************

from sage.graphs.graph import DiGraph
from sage.sets.set import Set
from mapping import Mapping
from mapping import Involution

class GeneralGraph(DiGraph):
    r"""
    General Graph: Is a tuple (E,V,s,t) where:
        E is a set representing the "Edges" of the 
        General Graph
        V is a set representing the "Vertices" of the 
        General Graph
        s is a maping from E into V, where s(e) = v 
        denotes that vertex
        v is the source of edge e.
        t is a mapping from E into V, where t(e) = v 
        denotes that vertex c is the terminus of edge e.

    Usage: G = GeneralGraph(E, s, t)
    Input: Edge Set E, source mapping s, and a 
    terminus mapping t
    Output: A General Graph G
    References: TODO
    Examples:
        sage: E = [ 'a','b','c','d']
        sage: s = Mapping({ 'a':1, 'b':2, 'c':2, 'd':3 })
        sage: t = Mapping({ 'a':2, 'b':1, 'c':3, 'd':2 })
        sage: G = GeneralGraph(E, s,t)
        sage: G
    
        A GeneralGraph on 3 verticies with edge set:
            {'a', 'c', 'b', 'd'}
        and source map:
            [('a', 1), ('c', 2), ('b', 2), ('d', 3)]
        and terminus map:
            [('a', 2), ('c', 3), ('b', 1), ('d', 2)]
        sage: G.edges()
            [(1, 2, '(a:1)'), (2, 1, '(b:2)'), (2, 3,\
            '(c:2)'), (3, 2, '(d:3)')]
        sage: G.edge_set()
            {'a', 'c', 'b', 'd'}
    """

    _s = None
    _t = None
    _E = None

    def __init__(self, E, s,t):
        self._E = E
        self._s = s
        self._t = t

        DiGraph.__init__(self, multiedges=True, \
        loops=True, weighted=False)
        
        for e in E:
            DiGraph.add_vertex(self,s(e))
        for e in E:
            DiGraph.add_edge(self, s(e),t(e), \
            '(%s:%s)'%(e,s(e)))
    def edge_set(self):
        r"""
        The domain of the source and terminus mappings 
        of the GeneralGraph

        Usage: G.edge_set()
        Input: None
        Output: The edge set of the General Graph G
        Examples:
            sage: E = [ 'a','b','c','d']
            sage: s = Mapping({ 'a':1, 'b':2, 'c':2, 'd':3 })
            sage: t = Mapping({ 'a':2, 'b':1, 'c':3, 'd':2 })
            sage: G = GeneralGraph(E, s,t)
            sage: G.edge_set()
                {'a', 'c', 'b', 'd'}
        """
        return Set(self._E)
    def _repr_(self):
        return \
        "A GeneralGraph on %s verticies with edge set:\ 
        "%self.num_verts() \
        +'\n' + \
             '\t' + "%s" %self.edge_set() + '\n' + \
            "and source map:" + '\n' \
            '\t' + "%s"%self._s.list_relation() + '\n' + \
            "and terminus map:" + '\n' + \
            '\t' + "%s"%self._t.list_relation()
    def source_mapping(self):
        r"""
        Usage: G.source_mapping()
        Input: None
        Output: Returns the source mapping of the 
        General Graph G
        Examples: TODO
        """
        return self._s
    def source(self, e):
        r"""
        Usage: G.source(e)
        Input: An member of the edge set of 
        the General Graph G
        Output: A member of the vertex set of 
        the General Graph G
        where the edge is meant to originate from
        Examples: TODO
        """
        return self._s(e)
    def terminus_mapping(self):
        r"""
        Usage: G.terminus_mapping()
        Input: None
        Output: Returns the terminus mapping of 
        the General Graph G
        Examples: TODO
        """
        return self._t
    def terminus(self, e):
        r"""
        Usage: G.terminus(e)
        Input: An member of the edge set of 
        the General Graph G
        Output: A member of the vertex set of 
        the General Graph G
        where the edge is meant to terminate
        Examples: TODO

        """
        return self._t(e)
    def concatination(self, H):
        r"""
        Usage: K = G.concatination(H)
        Input: A General Graph H
        Output: A General Graph K that is the graph 
        concatenation of
        G with H
        Examples: TODO
        """
        E = self.terminus_mapping().fiber_product(\
        H.source_mapping())
        s = Mapping( E, lambda x: self.source(x[0]))
        t = Mapping( E, lambda x: H.terminus(x[1]))
        return GeneralGraph(E, s,t)
class InvolutiveGeneralGraph(GeneralGraph):
    r"""
    Involutive General Graph: Is a tuple (E,V,s,r) where:
     E is the set of "edges" of the Involutive General Graph
     V is the set of "verticies" of the Involutive General Graph
     s is a mapping from E into V where s(e)=v denotes that the 
     vertex v is the source of edge e
     r is an involution from E into E where s(e) = e' denoted 
     that edge e is "connected" to edge e'.
    An Involutive General Graph is a General Graph with the
    terminus mapping being given by t(e) = s(r(e))

    Usage: G = InvolutiveGeneralGraph(E,s,r)
    Input: Edge Set E, source Mapping s and 
    Involution r
    Output: A Involutive General Graph G
    Refrences: TODO
    Examples: TODO

    """
    _E = None
    _s = None
    _r = None
    _t = None

    def __init__(self, E, s,r):
        t = s.composition(r)

        self._E = E
        self._s = s
        self._r = r
        self._t = t

        GeneralGraph.__init__(self, E, s, t)
        
    def _repr_(self):
        return GeneralGraph._repr_(self) + '\n' + \
        'With involution:' + '\n'  + \
        '\t' + '%s'%self._r.list_relation()

    def involution_mapping(self):
        r"""
        Usage: r = G.involution_mapping()
        Input: None
        Output: An involution on the Edges of G 
        that defines the connections 
        between the edges of G
        Examples: TODO
        """
        return self._r
    def adjacent_edge(self,e):
        r"""
        Usage: G.adjacent_edge()
        Input: An edge of the Involutive General Graph G
        Output: An edge of the Involutive General Graph G, 
        that is "adjacent" as defined by the 
        involution mapping.
        Examples: TODO

        """
        return self._r(e)
    def zig(self, H):
        r"""
        The "zig" product of Involutive General Graph G with the 
        Involutive General Graph H

        Usage: K = G.zig(H)         
        Input: An Involutive General Graph H
        Output: An Involutive General Graph 
        K that is the "zig" 
        product of G with H
        References: TODO
        Examples: TODO

        """
        E = [ (v,e) for v in self.vertices()\
        for e in H.edge_set() ]
        s = Mapping(E, lambda x:\
        (x[0],H.source(x[1])))
        r = Involution(E, lambda x:\
        ( x[0], H.adjacent_edge(x[1])))

        return InvolutiveGeneralGraph(E,s,r)
    def zag(self,H,f):
        r"""
        The "zag" product of the Involutive 
        General Graph G with the 
        Involutive General Graph H

        Usage: K = G.zag(H)
        Input: An Involutive General Graph H 
        and mapping from the edge set 
        of G onto the vertex set of H
        Output: An Involutive General Graph K 
        that is the "zag" product of 
        G with H
        References: TODO
        Examples: TODO
        """
        E = self.edge_set()
        
        r = self.involution_mapping()
        s = Mapping(E, lambda e: (self.source(e), f(e)))
        return InvolutiveGeneralGraph(E, s,r)
        
    def zig_zag(self,H,f):
        r"""
        The "zig-zag" product of the Involutive 
        General Graph G with the Involutive General Graph H

        Usage: K = G.zig_zag(H,f)
        Input: An Involutive General Graph H 
        and a mapping from the edge set of G onto 
        the vertex set of H
        Output: An Involutive General Graph K 
        that is the "zig-zag" product 
        of G with H
        References: TODO
        Examples: TODO 
        """

        K_zig = self.zig(H)
        K_zag = self.zag(H,f)


        # Extract all mappings/Sets needed from zig
        t_zig = K_zig.terminus_mapping()
        s_zig_inv = (K_zig.source_mapping()).inverse()
        r_zig = K_zig.involution_mapping()
        E = K_zig.edge_set()

        # Extract all mappings needed from zag
        t_zag = K_zag.terminus_mapping()
        s_zag_inv = (K_zag.source_mapping()).inverse()

        s = K_zig.source_mapping()
        r = Mapping(E, lambda e: \
        r_zig(s_zig_inv(t_zag(s_zag_inv(t_zig(e))))))

        return InvolutiveGeneralGraph(E,s,r)
\end{sageblock}

\section{General Set Function Object with Involution}
\begin{sageblock}
#****************************************************************
#  Copyright (C) 2008 David Monarres <dmmonarres@gmail.com>
#
#  Distributed under the terms of the 
#  GNU General Public License version 2 (GPLv2)
#
#  This code is distributed in the hope that it will be useful,
#  but WITHOUT ANY WARRANTY; without even the implied warranty of
#  MERCHANTABILITY or FITNESS FOR A PARTICULAR PURPOSE.  
#  See the GNU General Public License for more details.
#
#  The full text of the GPL is available at:
#
#  http://www.gnu.org/licenses/
#****************************************************************
from sage.structure.sage_object import SageObject
from sage.sets.set import Set

class Mapping(SageObject):
        r"""
        Basic finite function class
        USAGE: Mapping(*args, **kwds)
        INPUT:
               args - can be either of the following:
                      * A list of tuples or python dictinary
                      * A finite Set (or list) and a function 
                        to apply to the set 
        OUTPUT: A callable function object

        Examples:
        You can define the relation as a list of tuples
        
               sage: f = Mapping( [ (1,2), (2,3), (4,5) ] )
               sage: f(4)
               5
               sage: f(2)
               3
        Or as a python dictionary
               
               sage: f = Mapping( { 1:2, 4:5, 6:7} )
               sage: f(6)
               7
               sage: f(4)
               5
        Or as a set and a function
               sage: G = GF(11)
               sage: H = GF(7)
               sage: f = Mapping( Set(G), lambda x: H(x))
               sage: f(10)
               3
               sage: f(10)
               3
               sage: f
               A Mapping from {0, 1, 2, 3, 4, 5, 6, 7, 8, 9, 10} 
               to {0, 1, 2, 3, 4, 5, 6}
        """
        _f = None
        _domain = None
        _range = None
        _d = dict()
        def __init__(self, *args):
                if args == None:
                        pass
                elif len(args) == 1:
                        # Explicit definition of mapping
                        self._d = dict(args[0])
                        self._domain = Set( self._d.keys())
                        self._range = Set( self._d.values())
                elif len(args) == 2:
                        self._domain = Set(args[0])
                        self._f = args[1]
                        self._d = dict(  [(x, self._f(x)) \
                        for x in self._domain] )
                        self._range = Set(self._d.values())
                else:
                        raise NotImplementedError,\
                        "Mapping not implimented for \
                        that combination of arguments"

        def __call__(self,x):
                try: 
                        return self._d[x]
                except KeyError:
                        raise ValueError, \
                        "%s not in the domain of the mapping"%x
        def domain(self):
                r"""
                Domain of self mapping

                Input: None
                Output: The domain of the Mapping

                Examples:
                       sage: f = Mapping({ 1:2, 5:6, 8:10})
                       sage: f.domain()
                       {8, 1, 5}        
                """
                return self._domain

        def range(self):
                r"""
                Range of self mapping

                Input: None
                Output: The range of the Mapping

                Examples:
                       sage: f = Mapping({ 1:2, 5:6, 8:10})
                       sage: f.range()
                       {2, 6, 10}
                """
                return self._range

        def composition(self, g):
                r"""
                Compose two mappings

                Input: A mapping g
                Output: The composition mapping of self and g

                Examples:
                       sage: f = Mapping({ 1:2, 5:6, 8:10})
                       sage: g = Mapping( {  1:1, 2:5 } )
                       sage: h = f.composition(g)
                       sage: h
                       A Mapping from {1, 2} to {2, 6}
                       sage: h(1)
                       2
                       sage: h(2)
                       6
                """
                d = dict( [ (x, self(g(x)) ) for x in g.domain()])
                return Mapping(d)

        def pre_image(self,y):
                r"""
                The pre-image of an element
                
                Usage: m.preimage(y)
                Input: An element in the range of the mapping
                Output: The set x in the domain of f such that 
                f(x) == y and the empty set if none exists
                
                Examples:
                       sage: f = Mapping( {1:2, 2:2, 3:2})
                       sage: f.pre_image(2)
                       {1, 2, 3}
                       sage: f.pre_image(1)
                       {}
                """
                return Set( [ x for x in self.domain() if self(x) == y])

        def inverse(self):
                r"""
                Computes the inverse of the mapping f if one exists, 
                if not it raises a NotImplemented Error
                
                Usage: m.inverse()
                Input: None
                Output: 
                A Mapping g such that g(f(x)) = x\
                for all x in domain of f
                Examples: TODO
                """
                d = dict()
                for x in self.domain():
                        if d.has_key(self(x)):
                                raise NotImplementedError, \
                                "Mapping is not one -to -one"
                        else:
                                d[self(x)] = x
                return Mapping(d)

        def fiber_product(self,g):
                r"""
                The fibre porduct of two mappings
                INPUT: A Mapping g
                OUTPUT: (c, c') such that self(c) == g(c')
                REFRENCES: 
                EXAMPLES:
                  sage: f = Mapping( GF(9,'a'), lambda x: x^2 ) 
                  sage: g = Mapping(Gf(9,'a'), lambda x: x^3)
                  sage: f.fiber_product(g)
                  {(a, 2*a + 2), (a + 2, a + 1), (0, 0), (2*a + 2, 2),\
                   (2, 1), (2*a + 1, a + 1), (a + 1, 2),\
                   (1, 1), (2*a, 2*a + 2)}
                """
                A = self.range().intersection(g.range())
                ret = []
                for x in A:
                        [ ret.append( (y,z) ) for 
                        y in self.pre_image(x)\ 
                        for z in g.pre_image(x)]
                return Set(ret)
        def list_relation(self):
                r"""
                Returns the Mapping as a list of 
                tuples describing the relation
                Usage: m.list_relation()
                Input: None
                Output: List of tuples of the form 
                [ (x, f(x) for x in f.domain()]
                Examples: TODO
                """
                return [ (x, self(x)) for x in self.domain() ]
        def _repr_(self):
                return "A Mapping from 
                % s to %s"%(self.domain(),self.range())
        def _str_(self):
                return "%s"%self.list_relation()
class Involution(Mapping):
    r"""
    Involution: TODO
    Usage: m = Involution(list)
    Input: A list of one or two item lists that describe the disjoint 
    transpositions of the involution m
    Output: A callable mapping, m , such that m(m(x)) = x for all x in 
    the domain of the mapping
    Examples: 
        sage: m = Involution( [  [1,2], [4,5], [3]] )
        sage: m
        A Mapping from {1, 2, 3, 4, 5} to {1, 2, 3, 4, 5}
        sage: m(3)
        3
        sage: m(1)
        2
        sage: [ m(m(x)) for x in m.domain() ]
        [1, 2, 3, 4, 5]
    """
    def __init__(self, *args):
        if len(args) == 1:
            d = dict()
            # make sure that we are only dealing with transpositions
            for x in args[0]:
                if len(x) == 1:
                    # fixed value
                    if d.has_key(x[0]):
                        raise TypeError,\
                        "%s not disjoint from %s"%(x, [ x[0], d[x[0]]])
                    else:
                        d[x[0]] = x[0]
                elif len(x) == 2:
                    if x[0] == x[1]:
                        # Handle special case where fixed transposition 
                        # is given as [1,1]
                        d[x[0]] = x[1]
                    elif d.has_key(x[0]):
                        # allow a repeated transposition
                        if set(x) == set( [x[0], d[x[0]]]):
                            pass
                        else:
                            raise TypeError, \
                            "%s not disjoint from %s"%(x, [ x[0], d[x[0]]])
                    elif d.has_key(x[1]):
                        if set(x) == set( [x[1],d[x[1]]]):
                            pass
                        else:
                            raise TypeError, \
                            "%s not disjoint from %s"%(x, [ x[1], d[x[1]]])
                    else:
                        d[x[0]] = x[1]
                        d[x[1]] = x[0]
                else:
                    raise TypeError, "%s is not a transposition"%x
            Mapping.__init__(self, d)
        elif len(args) == 2:
            domain = args[0]
            f = args[1]

            for x in domain:
                if (f(f(x)) != x):
                    raise TypeError,\
                    'f does not define an involution on the given domain'
            Mapping.__init__(self, domain, f)
        else: 
            raise TypeError
\end{sageblock}
%%% Local Variables: 
%%% mode: latex
%%% TeX-master: "thesis"
%%% End: 
